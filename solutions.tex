\documentclass[a4paper,12pt,twoside]{book}
\usepackage[T2A]{fontenc}
\usepackage[utf8]{inputenc}
\usepackage{amsfonts}
\usepackage[russian,ukrainian]{babel}
\usepackage{amsthm}

\begin{document}
\chapter{Нормовані простори}
\section{Лінійні нормовані простори}
\textbf{1.1}. Довести, що в означенні лінійного нормованого простору аксіому $\|\mathbf{x}\| = 0$ тоді й лише тоді, коли $\mathbf{x} = \mathbf{0}$, можна замінити на аксіому: з $\|\mathbf{x}\| = 0$ слідує $\mathbf{x} = 0$
\begin{proof}
    Нам потрібно довести, що з того, що $\mathbf{x} = 0$ слідує $\|\mathbf{x}\| = 0$. Будемо від супротивного. Нехай $\mathbf{x} = 0$ та $\|\mathbf{x}\| = a \neq 0$. Тоді $\|\mathbf{0} + \mathbf{0}\| = 2\|\mathbf{0}\|  = 2a = a $. Це можливо лише тоді, коли $a = 0$.
\end{proof}
\textbf{1.2} Нехай $x_n, x, y_n, y \in X (n \in \mathbb{N})$. Довести що:
\begin{enumerate}
    \item якщо $x_n \rightarrow x$, то $x_n$ - обмежена послідовність
    \begin{proof}
        За означенням $\forall \epsilon > 0 \exists N(\epsilon): \forall n > N(\epsilon) \|x_n - x\| < \epsilon$. Оскільки послідовність необмежена, знайдемо таке $N > N(\epsilon)$, що $\|x_N\| > \|x\| + \epsilon + 1$. Візмемо $\epsilon = 1$. Тоді існує $N \in \mathbb{N}$ таке, що для всіх $n > N \|x - x_n\| < 1.$, з цього випливає, що $\|x_n\| \leq \|x\| + 1$. Нехай
        \begin{equation}
            M = \max \{\|x_1\|,\|x_2\|\, ..., \|x_N\|, \|x\| + 1\}
        \end{equation}
    Максимум існує оскільки множина скінченна. Тоді для всіх $n \in \mathbb{N}$ $\|x_n\|\leq M$.
    \end{proof}
    \item Якщо $x_n \rightarrow x$ та $\lambda_n \rightarrow \lambda, \lambda_n \in \mathbb{C}$, то $\lambda_n x_n \rightarrow \lambda x$.
    
    \begin{proof}
      $  \|\lambda_n x_n  - \lambda x\| = \|\lambda_n x_n -\lambda_n x + \lambda_n x - \lambda x\| = \|\lambda_n (x_n - x) + x(\lambda - \lambda_n)\| \leq |\lambda_n|\|x_n - x\| + \|x\||\lambda - \lambda_n| \rightarrow 0$. 
    \end{proof}
    \item якщо $x_n \rightarrow x$, то $\|x_n\| \rightarrow \|x\|$.
    \begin{proof}
        $|\|x_n\|| - \|x\|| \leq \|x - x_n\| \rightarrow 0$. Згідно оберненої нерівності трикутника.
    \end{proof}
    \item $x_n \rightarrow x$ та $\|x_n - y_n\| \rightarrow 0$, то $y_n \rightarrow x$.
    \begin{proof}
        $\|x - y_n\| = \|x - x_n + x_n - y_n\| \leq \|x - x_n\| + \|x_n - y_n\| \rightarrow 0$.
    \end{proof}
    \item якщо $x_n \rightarrow x$, то $\|x_n - y\| \rightarrow \|x - y\|$.
    \begin{proof}
        $|\|x_n - y\| - \|x - y\|| \leq \|x_ n - x\| \rightarrow 0$, згідно з оберненої рівності трикутника.
    \end{proof} 
    \item Якщо $x_n \rightarrow x, y_n \rightarrow y$, то $\|x_n - y_n\| \rightarrow \|x - y\|$.
    \begin{proof}
        $|\|x_n - y_n\| - \|x - y\|| \leq \|x_n - x + y_n - y\| \leq \|x_n - x\| \|y_n - y\| \rightarrow 0$.
    \end{proof}
\end{enumerate}
\end{document}